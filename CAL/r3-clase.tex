% Created 2017-11-11 sáb 17:48
\documentclass[11pt]{article}
\usepackage[utf8]{inputenc}
\usepackage[T1]{fontenc}
\usepackage{fixltx2e}
\usepackage{graphicx}
\usepackage{longtable}
\usepackage{float}
\usepackage{wrapfig}
\usepackage{rotating}
\usepackage[normalem]{ulem}
\usepackage{amsmath}
\usepackage{textcomp}
\usepackage{marvosym}
\usepackage{wasysym}
\usepackage{amssymb}
\usepackage{hyperref}
\tolerance=1000
\DeclareMathOperator{\R}{\mathbb{R}}
\setcounter{secnumdepth}{0}
\usepackage{amsthm}
\usepackage{amsmath}
\usepackage{tikz-cd}
\author{Ricardo Ruiz}
\date{\today}
\title{Relación 3 - Clase}
\hypersetup{
  pdfkeywords={},
  pdfsubject={},
  pdfcreator={Emacs 24.5.1 (Org mode 8.2.10)}}
\begin{document}

\maketitle

\section{Ejercicio 3}
\label{sec-1}

\subsection{Enunciado}
\label{sec-1-1}
Sea $x \in \R$. Prueba que el $\sup\{r \in \mathbb{Q}: r < x\} = x =
\inf\{s \in \mathbb{Q}: s > x\}$.  ¿Permanece válido este resultado si
se sustituye $\mathbb{Q}$ por un conjunto denso en $\R$?

\subsection{Demostración}
\label{sec-1-2}

Sea $A = \{r \in \mathbb{Q}: r < x \}$, está claro que $x$ es
mayorante de $A$.  Si $x$ no fuera el mínimo mayorante, existiría un
$y \in \R$, mayorante de $A$ tal que $y < x$.  Sin embargo, por la
densitud de $\mathbb{Q}$ en $\R$, existe $r \in \mathbb{Q}$, tal que
$y < r < x$.  Por tanto, $y$ no sería mayorante de $B$ y $x$ es
claramente el supremo.

Análogamente, sea $B = \{s \in \mathbb{Q}: s > x \}$, está claro que
$x$ es minorante de $B$. Si $x$ no fuera el máximo minorante,
existiría un $y \in \R$, minorante de $B$, tal que $y > x$.  Sin
embargo, por la densitud de $\mathbb{Q}$ en $\R$, existe $r \in
\mathbb{Q}$, tal que $y > r > x$.  Por tanto, $y$ no sería minorante
de $B$, y $x$ es claramente el ínfimo.

Permanecería válido pues la demostración no depende de $\mathbb{Q}$,
sino de su densitud en $\R$.
% Emacs 24.5.1 (Org mode 8.2.10)
\end{document}