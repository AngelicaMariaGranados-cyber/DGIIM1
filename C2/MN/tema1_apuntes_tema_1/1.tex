\ifdefined\COMPLETE
\else 
\def\COMPLETE{}
\documentclass[a4paper , 11pt, spanish ]{article}

% Codificación e idioma, para las tildes crucial
\usepackage[utf8]{inputenc}
\usepackage[spanish]{babel}
\usepackage[T1]{fontenc}
\usepackage{textcomp}

%paquetes para matemáticas
\usepackage{fullpage,mathpazo,amsfonts,nicefrac, amssymb , amsmath, amsbsy}
\usepackage{amsmath}
%paquetes para insertar gráficas (imágenes)
\usepackage{graphicx}

%paquete de enumerar, para realizar listas 
\usepackage{enumerate}

\usepackage{import}


%DEFINICICIONES 
\newcommand{\norm}[1]{\left\lVert#1\right\rVert}

\title{ Tema 1: Introducción a los problemas de análisis numérico}

\begin{document}
\fi

%\title{ Tema 1: Introducción a los problemas de análisis numérico}

\maketitle 
%%%%%%%%%%% ESPACIOS NORMADOS %%%%%%%%%%%%%%%%%%%%
\section{Espacios normados} 
%%%%%%%%%%%%%%%%%% definición de norma %%%%%%%%%%%%%
\subsection{Definición de norma} 
Sea E un espacio que admite una norma. \\
Se define como \textbf{ norma} de un espacio vectorial real o complejo E, a la plicación $\norm{.} : E \longrightarrow \mathbb{R} $ y que verifica las siguientes propiedades: \\
\begin{enumerate}
\item $ x \in E  \Rightarrow $ $\begin{Bmatrix}
	\norm{x} \ge 0\\
	\norm{x} = 0 \Leftrightarrow x = 0
\end{Bmatrix}$

\item  Desigualdad triangular: $\quad$ x, y $\in $ E $\norm{ x + y } \le \norm{x} + \norm{y}$

\item  $ x \in E, \quad \lambda \Rightarrow \norm{ \lambda x} = |\lambda | \norm{x}$
\end{enumerate}
%%%%%%%%%%%% ejemplos de norma %%%%%%%%%%
\subsection{Ejemplo de normas}
\subsubsection*{ Norma de vectores} 
Sea E = $\mathbb{R}^n ,\quad p \le 1$ 
\[\norm{x}_p :=  {(\sum_{j=1}^{N} |x_j|^p )} ^{\frac{1}{p}} \quad ( x \in \mathbb{R}^n)\]
p = 2 se conoce como la norma euclídea. 
\subsubsection*{Norma del máximo}
\[ \norm{x}_\infty := max _{ j=1...N} | x_j | \quad ( x \in \mathbb{R}^n)\]

Tambíen apliabla para la norma máximo de una función. 
\subsubsection*{Norma de Frobeius}
Usada en matrices
\[ \norm{A}_F := \sqrt{ \sum_{i=1}^{M} \sum_{j=1}^{N} a_{ij}^2} \quad (A \in \mathbb{R}^{M \times N}) \]

%%%%%%%%%%% aplicacione de norma %%%%%%%%%%%
\subsection{ Conceptos de análisis con normas }
Se define la distancia entre dos vectores x,y  de un espacion normado como:
$$ dist(x,y) := \norm{x -y} $$  
Gracias a esto podemos aplicar conceptos del cálculo con normas:

\subsubsection*{Convergencia , límites }  
Sea $ \{{ X_n\}}_{n \ge 1} $ una sucesión de E, se dice que esta sucesión converge a $ x_0 \in E $  
Si y sólo si, para toda $\varepsilon > 0 $  existe un $n_0 \in \mathbb{N}$ tal que: 
$$ \forall  n > n_0, \quad \norm{ x_n - x_0 } < \varepsilon $$   
De donde concluimos (siguiendo el hilo de nuestras amadas diapositivas de nuestro querido profesor, vamos, que leáis Payá si queréis enteraros) que:
$$ \lim_{n \to 	\infty}{ x_n} = x_0 	\quad \Leftrightarrow \quad \lim_{n \to 	\infty}{ \norm {x_n - x_0 } } = 0 $$

\subsubsection*{Continuidad de una aplicación en espacios normados} 
Sean X, Y dos espacios normados y f una aplicación de X a Y. Se dice que f es continua en $x_0 \in X$  si :
 $$\begin{Bmatrix}


	& x \in X & \\
	\forall \varepsilon > 0 , \quad \exists \delta > 0 \quad \Rightarrow & & 	\quad \Rightarrow  	\norm{f(x) - f(x_0)} <	\varepsilon  \\
	& \norm{x - x_0}=\delta & 
\end{Bmatrix}$$

\subsubsection*{Lema del sandwich}
$$ x \in \mathbb{R} \Rightarrow \norm{x}_\infty \le \norm{x}_1 \le N\norm{x}_\infty $$

Generalizando para cualquier norma: 
$$ x \in \mathbb{E} \Rightarrow c_1\norm{x} \le \norm{x}_* \le c_2\norm{x} $$

%%%%%%%%%%%%%%%% proposiciones y teoremas con normas %%%%%%
\subsection{ Proposiciones y teoremas con normas}
%%%%%%%%%%%%%%% teoremilla sin norma %%%%%%%%%
\subsubsection{ Teoremilla sin demostrar}
Todas las normas de un espacio finito dimensional son equivalentes. 

%%%%%%%%%%%%5 norma inducida %%%%%%%%%%%%%%%
\subsubsection{ Norma inducida en $\mathbb{R}^{ M\times N }$  por las normas en $\mathbb{R}^{ M}$ y  $\mathbb{R}^{ N }$}  
Sean M , N $\in \mathbb{N}$ y consideremos sendas normas en $\mathbb{R}^{ M}$ y $\mathbb{R}^{ N }$, que sin luga a ambiguedad notaremos como $\norm{.}$. Entonces al aplicación qu notaremos igualmente como $\norm{.}$ 
\[ \norm{A} := sup_{ x \in \mathbb{R}^N \\ \norm{x}= 1 } \norm{Ax} \quad A \in \mathbb{R}^{ M\times N } \]
\
\textbf{Demostración }:
\par 
En base a la definición de norma necesitaremos probar: \\
\begin{enumerate}
\item $ x \in E  \Rightarrow $ $\begin{Bmatrix}
	\norm{x} \ge 0\\
	\norm{x} = 0 \Leftrightarrow x = 0
\end{Bmatrix}$

\item  Desigualdad triangular: $\quad$ x, y $\in $ E $\norm{ x + y } \le \norm{x} + \norm{y}$

\item  $ x \in E, \quad \lambda \Rightarrow \norm{ \lambda x} = |\lambda | \norm{x}$
\end{enumerate}

Procedamos pues al espectáculo:

\begin{enumerate}
\item Demostrar la desigualdad estricta es trivial, ya que al realizar el producto de A por x se le aplicaría la normas en $\mathbb{R}^{ M}$ o $\mathbb{R}^{ N }$ para las que sí sabemos que se cumple, por tanto faltaría, faltaría comprobar si $ sup_{\norm{x}=1} \norm{Ax} = 0 \Leftrightarrow \norm{A}=0 $  entonces A=0.
\par
Procederemos por una doble desigualdad: 
\\ $\Leftarrow$
\\ Nuestras hipótesis son: 
\par 
\centering
$\begin{matrix}
	\norm{A} = 0 & y &\forall x \in \mathbb{R}^{N} \quad  Ax=0 \\
\end{matrix}$
\flushleft 
\par  
Si x=0 es trivialmente cierto, veamos que ocuerre en caso contrario: \\
$$ Ax = \norm{x} A \frac{x}{\norm{x}} = \norm{x}0 = 0$$  
Para la primera igualdad hemos dividido y multiplicado por la norma de x, en la segunda que un vector dividido por la norma de otro es otro vector y para la tercera la hipótesis, que la matriz por un vector es igual a 0. 
\\ De la igualdad anterior se deduce que $ sup_{\norm{x}=1} \norm{Ax} = 0 \Leftrightarrow \norm{A}=0 $

\par 
$\Rightarrow$
\par
Ahora solo nos faltaríacomprobar que para un vector cualquiera el producto con A sólo será 0 si A es 0, ya que para que el supremo sea 0, todos los vectores tienen que ser 0:

Consideramos la aplicación lineal :\\
\begin{tabular}{l}
	$t:  \mathbb{R}^{ N } \longrightarrow   \mathbb{R}^{M} $ \\
	$t: x \longrightarrow Ax $
\end{tabular}


 
La matiz de esta aplicación será A, ahora se tiene que $\forall x \in \mathbb{R}^{N} $ se cumple que  Ax = 0 entonces t(x) = 0 y esto solo se cumple para la aplicación identicamente nula, con lo cual A=0.  

\item Aquí basta con aplicar las propiedades de las matrices: 
$$ (A + B)x = Ax + Bx \quad \forall A \in A,B \in \mathbb{R}^{M \times N}  \quad x \in \mathbb{R}^{N}$$ 
Como sabesmso que $\mathbb{R}^{N}$ en normado en $\norm{.}$ queda demostrada la desigualdad triangualar en matrices. 

\item Partiendo de $\lambda$ es una constante perteneciente a números reales, es sabido que  $ sup \{ \norm{ \lambda Ax } \} =  |\lambda | sup\{\norm{Ax} \}$ que es lo que pretendiamos demostrar.  

\end{enumerate} 
%%%%%%%%%%%%%%%%%%% proposición consecuencia de la norma inducidad &&&&&&&& 
\subsubsection{ Consecencia de la proposición }

$$ Si A \in  \mathbb{R}^{M \times N}  \Rightarrow  \norm{A}= sup_{\norm{x}=1 , x \not = 0 }\frac{ \norm {Ax}}{ \norm{x}}  $$ 
Y en particular 
$$ x \in \mathbb{R}^n \Rightarrow \norm{Ax} \le \norm{A} \norm{x} $$ 
\par 
\textbf{Demostración} 
\par 
La primera igualdad es trivial si tenemos en cuenta la demostración anterior y que la $\norm{x}=1$ . 
\par 
Para la segunda igualdad probaremos primero que: 
$$ \norm{A} = sup \{ \frac{ \norm{Ax}}{\norm{x}} : x \in \mathbb{R}^n \quad y \quad  x \not = 0\}$$
\par Para ello primero tengamos presente la defincnición de la norma de A: 
\\ \[ \norm{A} := sup \{ \norm{Ax} :  x \in \mathbb{R}^N \\ \norm{x}= 1 \} \]  
\\   
Procederemos a probar por una doble inclusión que ambos conjuntos son iguales y que por tanto su supremo $\norm{A}$ lo es también. 
\par 
 \[ \{ \norm{Ax} :  x \in \mathbb{R}^N \\ \norm{x}= 1 \}  = \{ \frac{ \norm{Ax}}{\norm{x}} : x \in \mathbb{R}^n \quad y \quad  x \not = 0\} \]  
$\subseteq $
\\ 
$$ \frac{ \norm{Ax}}{\norm{x}}  = \norm{A \frac{x}{\norm{x}}} $$ 
En base a que el miembro de la izquierda es un elemento del conjunto de la izquiera y que la $\norm{x} = 1$ lo hemos escrito de manera equivalente de una forma que resulta evidente que pertene al conjunto de la derecha que es lo que pretendiamos demostrar. 
\\
$\supseteq$ 
\\
Observemos que si $\norm{x} = 1 $ para un vector del conjunto de la derecha, este pertenece al conjunto de la izquierda, que es lo que pretendíamos probar. 

\end{document}