\documentclass[a4paper , 11pt, spanish ]{article}

% Codificación e idioma, para las tildes crucial
\usepackage[utf8]{inputenc}
\usepackage[spanish]{babel}
\usepackage[T1]{fontenc}
\usepackage{textcomp}

%paquetes para matemáticas
\usepackage{fullpage,mathpazo,amsfonts,nicefrac, amssymb , amsmath, amsbsy}
\usepackage{amsmath}
%paquetes para insertar gráficas (imágenes)
\usepackage{graphicx}

%paquete de enumerar, para realizar listas 
\usepackage{enumerate}

\usepackage{import}


%DEFINICICIONES 

\title{Minimización de la desviación absoluta de la media respecto de la mediana}
\author{Blanca Cano Camarero}

\begin{document}
\maketitle
Se quiere probar que la desviación absoluta de la media respecto de la mediana es mínima, es decir, que $D_x$ alcanza un mínimo en X=Me:  
$$ D_{X} = \frac{\sum_{i = 1}^{k} |x_i - X| n_i}{N}$$ 

\textbf{Notación y conceptos usados } 
\par
\begin{itemize} 
	\item Se parte de que en una población de tamaño N, se ha observado la variable estadística ordinal X, que ha presentado k modalidades posibles con una distribución de frecuencias $\{x_i , n_i\}_{i=1...k}$ ...
	\item Prescindiré de la frecuencia absoluta y utilizaré directamente los datos ordenados, es decir que $x_i$ es el dato que ocupa la posición i-ésima en la distribución ordenada.
	\item Trabajaremos con cuantiles, siendo $x_\alpha$ el cuantil de orden $\alpha$, y tengamos presente que la mediana se da cuando $\alpha = 0.5 $.
	
\end{itemize}
\textbf{Demostración}
\par
Observemos que como conocemos el valor intermedio $X_\alpha$ podemos redefinir la desviación absoluta media respecto a $X_\alpha$ sin utilizar valores absolutos y en base al total de la población: 
\begin{equation}
	D_{X_\alpha} = \frac{1}{N}\Big\{ \sum_{i=1}^{\alpha N} \big(  X_\alpha - x_i \big) +  \sum_{i=\alpha N+1}^{N} \big( x_i - X_\alpha\big) \Big\}
\end{equation}
Si ahora derivamos (1) respecto de $ X_\alpha$    resulta:
$$ D_{X_\alpha}' = \frac{\alpha N}{N} - \frac{N- \alpha N}{N} = \alpha - (1-\alpha )$$
Esta función alcanza un punto crítico en $\alpha = 0.5 $. Para valores menores que 0,5 vemos que la derivada es negativa, mientras que para valores mayores es positiva, por tanto en $\alpha = 0.5 $ la función alcanza un mínimo, lo que es equivalente a que la función se minimice respecto a la mediana que es lo que queríamos probar. 
\end{document}