
\documentclass[a4paper , 11pt, spanish ]{article}

% Codificación e idioma, para las tildes crucial
\usepackage[utf8]{inputenc}
\usepackage[spanish]{babel}
\usepackage[T1]{fontenc}
\usepackage{textcomp}

%paquetes para matemáticas
\usepackage{fullpage,mathpazo,amsfonts,nicefrac, amssymb , amsmath, amsbsy}
\usepackage{yhmath} %números periódicos 4,\wideparen{9}

%paquetes para insertar gráficas (imágenes)
\usepackage{graphicx}

%paquete de enumerar, para realizar listas 
\usepackage{enumerate}

\usepackage{endnotes} %para notas a pie de sección

\graphicspath{ {img/r1/} } %path donde se encentran las imágenes

\title{Cuantil de orden $\alpha$}
\author{Blanca Cano Camarero}

\begin{document}
\maketitle

\section*{Introducción}
Partiendo de definiciones básicas: \\ 
De una población de tamaño n se ha observado una variable estadística $X$, que ha presentado $k$ modalidades posibles con una distribución de frecuencia \{$x_{i} , n_{i}$\}.  \par 
Si en particular no hallamos en una escala de medida ordenable, es decir ordinal, de intervalo o de razón. Siendo $\alpha$ un número real mayor que 0 y menor que 1,  y supuestos ordenados por valor del carácter a los individuos, nos puede surgir la preguna de cuál de tales valores deja por debajo a $\alpha n$  individuos, a este valor de medida de promedio se le conoce como \textbf{Cuantil de orden $\alpha$ }, $C_{\alpha}$.  
\par

%%%%%%%%%%%%%%%%%%%%%%%%%%%% VARIABLES DISCRETAS %%%%%%%%%%%%%%%%%%%%%%%%%
\section*{$C_{ \alpha}$ para variables discretas } 
En base a la definición dada , propongo para el cálculo de esta al primer valor de la variable cuya frecuencia absoluta acumulada sea mayor o igual a $\alpha n$.
\begin{equation}
 C_{ \alpha } = min \{ x_{i} : \alpha n \leq N_{i}\} \quad \forall i \in \{1...k\}
\end{equation}
\par 
Aunque en la definición dada se deja claro qué dato tomar a la hora de la práctica, procedo a concretar la casuística para una mayor claridad:
\begin{itemize}
\item Para el caso en el que el inverso de $\alpha$ no es divisor de $n$, el dato resultante del cálculo no será un número natural, o en el que $\alpha n $  no coincide con ninguan frecuencia absoluta acumulada,bastará con coger la primera modalidad cuya frecuencia absoluta acumulada supere tal posición.   \\
Ejemplo: \\
En una población de 7 individuos ha sido observada una variable ordinal concreta que ha presentado 3 modalidades distintas; que denotaremos con los tres primeros números naturales, con la siguiente distribución de frecuencia.
\begin{table}[h]
\begin{tabular}{|c|c|c|}
\hline
$x_{i}$ & $n_{i}$ & $N_{i}$\\
\hline
1 & 2 & 2 \\
2 & 1 & 3 \\
3 & 4 &  7 \\
\hline
\end{tabular}
\centering
\begin{tabular}{l}
$n_{i}$: frecuencias absolutas \\
$N_{i}$: frecuencias absolutas acumuladas \\
\end{tabular}
\end{table}
\par
 Se quiere calcular el cuantil de orden $ \frac{1}{3}$. \\
 Para ello, procederíamos de la siguiente manera: 
 \begin{enumerate}
 \item Cálculo de $ \alpha n$, en nuestro caso 2,$\wideparen{3}$ .
 \item Selección del primer $N_{i}$ que supere a $ \alpha n$, en nuestro caso $x_{2}$ que es 2.
\end{enumerate}
\item En el caso de que $ \alpha n$ sea igual que alguna de nuestras frecuencias absolutas acumuladas bastará con coger esa modalidad. 
\end{itemize}

%%%%%%%%%%%%%%%%%%%%%%%%% VARIABLES CONTINUAS %%%%%%%%%%%%%%%%%%%%%%%%%%%%%
\section*{$C_{ \alpha}$ para variables continuas } 
Una vez definido el método para variables discretas, podemos definir el de variables continuas sustituyendo las modalidad por intervalos de modalidades: 
\par 
De una población de tamaño n se ha observado una variable estadística $X$, que ha presentado $k$ intervalos posibles con una distribución de frecuencia \{$ I_{i} , n_{i}$\} en el intervalo $I_{i}$ definidido por ]$e_{i-1}$ , $e_{i}$ ]\endnote{ En caso de tratarse del intervalo $I_{i}$, tomar su intervalo cerrado por la izquierda.}	 $\forall i \in \{1...k\}$ .
\par
Como criterio se selección del itervalo podríamos tomar la misma fórmula con la misma sustitución: 

\begin{equation}
 \mathbb{I}= min \{ I{i} : \alpha n \leq N_{i}\} \quad \forall i \in \{1...k\}
\end{equation}
\par 
Sin embargo, lo interesante en las variables continuas será el cálculo del valor del intervalo $\mathbb{I} $ representante del cuantil, ya que si contáramos con todos los datos ordenados prescindiendo del intervalo, este caso se reduciría al de variable discreta, así que suponiendo que desconocemos la función de distribución de frecuencias absolutas acumuladas 
\endnote{ 
En todos los casos se podría trabajar con frecuencias realivas acumuladas, la única variación a tener en cuenta en las fórmulas presentadas, es el sustituir en las ecuaciones el $\alpha n$ por $ \alpha$
} 
de los datos en el intervalo $\mathbb{I} $ : ($e_{i}$ $e_{i+1}$] , tomameremos como ella la media 
\endnote
{ 
He definido la función media como la $\mathbb{F}$ de la ecuación 3, ya que puede haber infinitas funciones continuas que tomen 2 valores en un intervalo cerrado, pero si las visualizamos en un espacio $ \mathbb{R}^2$, la función que divide el área resultante en dos partes iguales es $\mathbb{F}$.
} 
de todas las posibles funciones, es decir, la lineal cuya pendiente es: $$ m= \frac{n_{i}}{ e_{i}-e_{i-1} }$$ 
\par y su ecuación \endnote{En caso de ser $\mathbb{I} =I_{1}$, tómese $n_{i-1} = 0 $, esto no es una convección si no que atendiendo a la deficinición de frecuencia absoluta tiene todo el sentido.}: 
\begin{equation}
 \mathbb{F}(x) = N_{i-1} + m ( x-e_{i-1}) \quad =  \quad N_{i-1} + \frac{n_{i}}{ e_{i}-e_{i-1} } ( x-e_{i-1}) \quad \forall x \in ]e_{i-1}, e_{i} ] 
\end{equation}

Por lo tanto una vez definida $\mathbb{F}$, podemos hallar el cuantil como el valor x tal que $\mathbb{F}$(x) sea igual a $ \alpha n$. 
\\ Despejemos de de la ecuación 3: 
$$\alpha n = \mathbb{F}(x) $$
$$ \alpha n = N_{i-1} + m ( x-e_{i-1})$$
$$ x = \frac{\alpha n - N_{i-1}}{ m} + e_{i-1} $$
Resultando que el valor de nuestro cuantil de orden $\alpha$ $x$ se obtiene de la siguiente fórmula, una vez que ya se conoce el intervalo: 

\begin{equation}
x = \frac{\alpha n - N_{i-1}}{ \frac{n_{i}}{ e_{i}-e_{i-1} }} + e_{i-1}
\end{equation}


%%%%%%%%%%%%%%%%%%%%%%%%%%%%%%%%%%%%%%%%%%%%%%%%%%%%%%%%%%%%%%%%%%%%%%%
\section*{ Por qué $\alpha $ se define en un intervalo ]0,1[ }  
Habrá usted notado que $\alpha $ se definió como un valor perteneciente a $\mathbb{R}$ mayor estricto que 0 y menor estricto que 1, procedo a explicar el porqué:
\par Estudiar el cuantil de orden 0 carece de interés ya que todos los datos serán superiores a él y bastará con coger el primero.  
\par Al igual que resulta ilógico estudiar el cuantil de orden 1, ya que carece de sentido preguntarse qué valor deja por debajos a todos los de la población.

%%%%%%%%%%%%%%%%%%%%%%%%%%%%%%%%%%%%%%%%%%%%%%%%%%%%%%%%%%
\section*{Valor que deja por arriba a $\alpha n $ valores de una población} 

Aplicando los mecanismo anteriores podríamos resolver el problema, o ordenando en orden de mayor a menor la tabla de estadística o si mantenemos el orden de alterando en las definiciones  y ecuaciones el $\leq$ por $\geq$.

%%%%%%%%%%%%%%%%%%%%%%%%%% NOTAS &&&&&&&&&&&&&&&&&&&&
\theendnotes 
\end{document}
