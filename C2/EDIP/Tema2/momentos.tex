\documentclass[a4paper , 11pt, spanish ]{article}

\documentclass{article}
\usepackage[spanish]{babel}
\usepackage[utf8]{inputenc}
\usepackage{amsmath}
\usepackage{amsthm}
\usepackage{mathtools}
\usepackage{graphicx}
\usepackage{mdframed}
\usepackage{lipsum}
\usepackage[makeroom]{cancel}

\theoremstyle{definition}
\newtheorem{definition}{Definición}[section]
\newmdtheoremenv{theorem}{\textit{Teorema}}
\newtheorem{corollary}{Corolario}[theorem]
\newtheorem{lemma}[theorem]{Lema}

\title{Momentos conjuntos no centrales en función de los centrales}
\author{Ricardo Ruiz Fernández de Alba}

\begin{document}
\maketitle

\begin{theorem}
Los momentos conjuntos no centrales pueden expresarse en función de los
centrales a partir de la siguiente relación
$$
\mu_{rs} = \sum_{t=0}^r \sum _{q=0}^s
    (-1)^{t+q} \binom{r}{t} \binom{s}{q} m_{10}^t m_{01}^q m_{r-t,s-q}
$$
\end{theorem}

\begin{proof}
\begin{align*}
\mu_{rs} = \sum_{i=1}^k \sum_{j=1}^p f_{ij} (x_i - m_{10})^t (y_j - m_{01})^s = \\
= \sum_{i=1}^k \sum_{j=1}^p f_{ij}
            \sum_{t=0}^r (-1)^t \binom{r}{t} m_{10}^t x_i^{r-t}
            \sum_{q=0}^s (-1)^q \binom{s}{q} m_{10}^q y_j^{r-q}  = \\
= \sum_{i=1}^k \sum_{j=1}^p f_{ij}
            \sum_{t=0}^r \sum_{q=0}^s
            (-1)^t \binom{r}{t} m_{10}^t x_i^{r-t}
            (-1)^q \binom{s}{q} m_{10}^q y_j^{r-q}  = \\
= \sum_{i=1}^k \sum_{j=1}^p  \sum_{t=0}^r \sum_{q=0}^s
            (-1)^t \binom{r}{t} m_{10}^t x_i^{r-t}
            (-1)^q \binom{s}{q} m_{10}^q y_j^{r-q}
            f_{ij} = \\
= \sum_{t=0}^r \sum_{q=0}^s \sum_{i=1}^k \sum_{j=1}^p
            (-1)^t \binom{r}{t} m_{10}^t x_i^{r-t}
            (-1)^q \binom{s}{q} m_{10}^q y_j^{r-q}
            f_{ij} = \\
= \sum_{t=0}^r \sum_{q=0}^s
          (-1)^{t+q} \binom{r}{t} \binom{s}{q} m_{10}^t m_{10}^q
          \left[\sum_{i=1}^k \sum_{j=1}^p x_i^{r-t} y_j^{r-q} f_{ij}\right] = \\
= \sum_{t=0}^r \sum_{q=0}^s
          (-1)^{t+q} \binom{r}{t} \binom{s}{q} m_{10}^t m_{10}^q m_{r-t,s-q}
\end{align*}
\end{proof}

\newpage
\begin{corollary}[La covarianza es la media del producto menos el pruducto de las medias]
$$
\sigma_{xy} = m_{11} - m_{10}m_{01}
$$
\end{corollary}

\begin{proof}
Como $\sigma_{xy} = \mu_{11}$, se tiene en particular que
\begin{align*}
\sigma_{xy} = \mu_{11} = \sum_{t=0}^1 \sum _{q=0}^1
    {(-1)}^{t+q} \binom{1}{t} \binom{1}{q} m_{10}^t m_{01}^q m_{1-t,1-q} = \\
= (-1)^0 \binom{1}{0} \binom{1}{0} m_{10}^0 m_{01}^0 m_{11}
  - \binom{1}{0} \binom{1}{1} m_{10}^0 m_{01}m_{10}  \\
  - \binom{1}{1} \binom{1}{0} m_{10}m_{01}^0 m_{01}
  + \binom{1}{1} \binom{1}{1} m_{10}m_{01}{m_00} = \\
= \cancel{(-1)^0} \cancel{\binom{1}{0}\binom{1}{0}} \cancel{m_{10}^0 m_{01}^0}
  m_{11}
  - \cancel{\binom{1}{0}\binom{1}{1}}\cancel{m_{10}^0} m_{01}m_{10} \\
  - \cancel{\binom{1}{1} \binom{1}{0}} m_{10}\cancel{m_{01}^0} m_{01}
   + \cancel{\binom{1}{1} \binom{1}{1}} m_{10}m_{01}\cancel{m_00} = \\
   = m_{11} - 2m_{01}m_{10} + m_{10}m{01} = \\
   = m_{11} - m_{10}m_{01} = \
   =  \sum_{i=1}^k \sum_{j=1}^p f_{ij} x_i y_j - \overline{x}\overline{y}
\end{align*}
\end{proof}

\end{document}
