\documentclass[a4paper , 11pt, spanish ]{article}

\documentclass{article}
\usepackage[spanish]{babel}
\usepackage[utf8]{inputenc}
\usepackage{amsmath}
\usepackage{amsthm}
\usepackage{mathtools}
\usepackage{graphicx}
\usepackage{mdframed}
\usepackage{lipsum}
\usepackage[makeroom]{cancel}

\theoremstyle{definition}
\newtheorem{definition}{Definición}[section]
\newmdtheoremenv{theorem}{\textit{Teorema}}
\newtheorem{corollary}{Corolario}[theorem]
\newtheorem{lemma}[theorem]{Lema}

\title{Media de los residuos de funciones polinómicas.}
\author{Ricardo Ruiz Fernández de Alba}

\begin{document}
\maketitle

\begin{lemma}
Siendo $y = p(x) = a_0 + \cdots + a_nx^n$, su media aritmética
corresponderá a los productos de los coeficiente z-ésimos del polinomio y
momento centrales de orden z, 0.
$$
\overline{y} = \sum_{i=1}^k\sum_{j=1}^p f_{ij} (a_0 + a_1x + \ldots + a_nx^n) =
\sum_{z=0}^n a_z m_{z0}
$$
\end{lemma}

\begin{proof}

\begin{align*}
\sum_{i=1}^k\sum_{j=1}^p f_{ij} (a_0 + a_1x_i + \ldots + a_nx_i^n) = \\
= a_0 \sum_{i=1}^k\sum_{j=1}^p f_{ij} + a_1 \sum_{i=1}^k \sum_{j=1}^p f_{ij}x_i
+ \cdots + a_n \sum_{i=1}^k \sum_{k=1}^p f_{ij}x_i^n = \\
= a_0 m_{00} + a_1 m_{10} + \cdots + a_n m_{n0} = \sum_{z=0} a_z m_{z0j}
\end{align*}

\end{proof}

\begin{corollary}
Si se somete una variable X a una transformación lineal afín, la media aritmética
de la nueva variable es la imagen de la media de X por la misma transformación.
$$
y = ax+b \Rightarrow  \overline{y} = a\overline{x} + b
$$
\end{corollary}
\begin{proof}
En particular para $p(x) = ax + b$, se tiene que
$$
\overline{y} = am_{01} + bm_{00} = a\overline{x} + b
$$
\end{proof}

\newpage

\begin{theorem}
La media de los residuos de funciones polinómicas es nula
$$
\overline{e} = \sum_{i=1}^k \sum_{j=1}^p f_{ij}e_ij =
\sum_{i=1}^k \sum_{j=1}^p  f_{ij} \left[y_j - p(x)\right]= 0
$$
\end{theorem}

\begin{proof}
\begin{align*}
\sum_{i=1}^k \sum_{j=1}^p  f_{ij}\left[y_j - (a_0 + a_1x + \cdots + a_nx^n )\right] = \\
 = \sum_{j=1}^p \sum_{i=1}^k (f_{ij}y_j -  f_{ij}a_0 -  \cdots - f_{ij}a_nx^n) = \\
 = \sum_{j=1}^p \left[\sum_{i=1}^k f_{ij}y_j - a_0\sum_{i=1} f_{ij} -
                      \cdots - a_n\sum_{i=1} f_{ij}x^n\right] = \\
 = \sum_{i=1}^k\sum_{j=1}^p f_{ij}y_j - a_0\sum_{i=1}^k\sum_{j=1}^p f_{ij} -
                      \cdots - a_n\SUm_{i=1}^k\sum_{j=1}^p f_{ij}x^ = \\
= \overline{y} -(a_0m_{00} + \cdots + a_nm_{n0}) = \\
= \overline{y} - \overline{y} = 0
\end{align*}
\end{proof}


\end{document}
