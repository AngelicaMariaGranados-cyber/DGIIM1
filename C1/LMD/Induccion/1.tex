% Created 2017-10-10 mar 23:20
\documentclass[11pt]{article}
\usepackage[utf8]{inputenc}
\usepackage[T1]{fontenc}
\usepackage{fixltx2e}
\usepackage{graphicx}
\usepackage{longtable}
\usepackage{float}
\usepackage{wrapfig}
\usepackage{rotating}
\usepackage[normalem]{ulem}
\usepackage{amsmath}
\usepackage{textcomp}
\usepackage{marvosym}
\usepackage{wasysym}
\usepackage{amssymb}
\usepackage{hyperref}
\tolerance=1000
\setcounter{secnumdepth}{0}
\usepackage{amsthm}
\usepackage{amsmath}
\usepackage{tikz-cd}
\author{Ricardo Ruiz}
\date{\today}
\title{Ejercicio 1}
\hypersetup{
  pdfkeywords={},
  pdfsubject={},
  pdfcreator={Emacs 24.5.1 (Org mode 8.2.10)}}
\begin{document}

\maketitle

\section{Enunciado}
\label{sec-1}

Demuestra por el principio de inducción que $P(n)$ se 
verifica $\forall n \in \mathbb{N}, n \neq 0$

\[
P(n) = \prod_{k=1}^n \left(1 - \frac{1}{(k+1)^2}\right)
= \frac{n+2}{2n+1} \quad \quad n > 0
\]


\subsection{Demostración}
\label{sec-1-1}

\subsubsection{Paso base}
\label{sec-1-1-1}

\[
P(1) = \prod_{k=1}^1 \left(1 - \frac{1}{(k+1)^2}\right) 
= 1 - \frac{1}{(1+1)^2} = 1 - \frac{1}{4} = \frac{3}{4}
\]

\[
P(1) = \frac{1+2}{2\cdot1+2} = \frac{3}{4}
\]

Por lo que $P(1)$ es cierta para $n=1$

\subsubsection{Paso de inducción.}
\label{sec-1-1-2}
Supongamos que $1\leq k$ y que $P(k)$ es cierta (Hipótesis de
Inducción).

Demostremos que de ello se deduce que $P(n+1)$ es cierta

\begin{eqnarray*}
\prod_{k=1}^{n+1} \left(1 - \frac{1}{(k+1)^2}\right) = 
\frac{n+1+2}{2(n+1)+1} = \frac{n+3}{2n+2+2} = \frac{n+3}{2n+4}
\end{eqnarray*}

Así pues:

\begin{align*}
\prod_{k=1}^{n+1} \left(1 - \frac{1}{(k+1)^2}\right)
= \left[\prod_{k=1}^n \left(1 - \frac{1}{(k+1)^2}\right)\right]
\cdot \left(1-\frac{1}{(n+1+1)^2}\right) = \\
= \frac{n+2}{2n+2} \cdot \frac{(n+2)^2-1}{(n+2)^2}
= \frac{(n+2)^2-1}{(2n+2)(n+2)}
= \frac{n^2+4+4n-1}{2n^2+2n+4n+4} = \\
= \frac{n^2+4n+3}{2n^2+6n+4}
= \frac{(n+1)(n+3)}{2(n+1)(n+2)}
= \frac{n+3}{2n+4}
\end{align*}


Por lo que, por el principio de inducción finita, $P(n)$
es cierta para todo $n>0$
% Emacs 24.5.1 (Org mode 8.2.10)
\end{document}