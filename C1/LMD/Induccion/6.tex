% Created 2017-10-07 s�b 19:53
\documentclass[11pt]{article}
\usepackage[latin1]{inputenc}
\usepackage[spanis]{babel}
\usepackage[T1]{fontenc}
\usepackage{amsmath}
\usepackage{amssymb}
\usepackage{amsthm}


\author{Blanca Cano}
\date{\today}
\title{Ejercicio 6}

\begin{document}

\maketitle

\section{proposici�n}
\label{sec-1}
Todo n�mero natural mayor que 1 es divisible por al menos
un n�mero primo.

\section{Demostraci�n}
\label{sec-2}

Definamos la aplicaci�n f(x,y) \r
Sea $A$ un conjunto que te devuelve los m�ltiplos de $x$

$$
A(x) = \left\{x \in \omega \geq x \in  \left\{f(x, a) \quad \forall a \in \omega \right\} a > 1  \right\}
$$


$P$ es un conjunto de n�meros primos que luego definiremos,
pero para que tenga sentido: $P=\emptyset$

\subsection{Observaci�n 1}
\label{sec-2-1}

Todo elemento contenido en $C$ es como m�nimo m�ltiplo del
primer elemento del conjunto, y, por definici�n ese primer
elemento es siempre m�ltiplo.

Sea A un conjunto definido por inducci�n tal que

$$
A(0) = {\emptyset}
A(1) = C(2)
A(2) = C(2) \bigcup C(3)
$$
y, generalizando
$$
A(n) = C(n+1) \cup (n+2)
$$

Definimos P, el conjunto de los n�meros primos

$$
P = \left\{x: x > 1 \quad \forall x \in \omega: x \notin A(x-1) \right\}
$$

Para todo elemento perteneciente a P, su �nico m�ltiplo $> 1$,
ser� �l mismo.

El resto de elementos $N = \omega -P \quad N \subset A$ y
como $A$ est� construido a partir de $C$, se cumple 
\texttt{observaci�n anterior 1}, luego todo $n \in N$ tiene al menos
un m�ltiplo, el primero de $C$ al que pertenezca, que, por
definici�n es primo.

Luego todo n�mero natural mayor que 1 es m�ltiplo al menos
de un n�mero primo.
% Emacs 24.5.1 (Org mode 8.2.10)
\end{document}